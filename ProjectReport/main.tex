\documentclass{article}
\usepackage[utf8]{inputenc}

\title{Deep Learning Sentiment Analysis on Twitter Tweets}
\author{Md. Sakib Ibne Farhad, 012192004 \\
		Shakil Ahmed, 012203048 \\		
		CSE6211 (M): Deep Learning}
\date{\today}

\begin{document}

\maketitle
\begin{abstract}
    Here put a summary of your work. You have to write it in three parts in a single paragraph. First the background of the problem. Second, what you did in the project. Third what results you have got. (\textbf{3 marks})
\end{abstract}

\section{Introduction} 
The introduction will be very important part of your project report. It will have three paragraphs. In the first paragraph, you have to say what problem you are solving. Provide a basic background of the problem. Why this problem is important to solve and what impact it might have. (\textbf{1 mark})

In the second paragraph, you provide a literature review on the work. What are the similar works? Here you have to comment of their effectiveness on solving the same problem. Try to tell about the strengths and limitations of their work. You have to find out at least 5 relevant work of your problem. And must cite them. \cite{jones2004introduction}. (\textbf{5 marks})

In the third paragraph, you establish why your work is important. Referring to the second paragraph where you have discussed about the literature of the work, here you have to justify the need of your own work. (\textbf{1 mark})

\subsection{Our Contribution}
In this section you use bullet points or enumerations to say what are the basic contributions of your project. (\textbf{1 mark})

\section{Material and Methods}
First, provide a block diagram of your work showing a graphical abstract of your work. You have to describe the figure here. (\textbf{2 marks})

\subsection{Dataset Description}
Describe the datasets that you have used/collected. You must provide url / citations to the dataset if not collected by you. Use a table to show the properties of the dataset. If you have collected your own dataset, you must put it in a github repository and put the link in this section. If required put histograms, density distribution or samples from your data here also state why you have chosen the data. Describe all the features. (\textbf{3 marks})

\subsection{Neural Network Models / Algorithms Used}
Describe the details of the Neural Network models or architectures that you have used. You have to use multiple models. For each model, put separate diagrams and describe them in the text. Why you have chosen them? What were the parameters and hyperparameters that you have used? Why you have used them? (\textbf{3 marks})

\subsection{Performance Evaluation}
Here you have to say about the error function that you have used and how you have divided the dataset for train, validation and test purpose. (\textbf{1 mark})

\section{Experimental Analysis}
Here provide tables and graphs and describe them. For each model, there must be results with multiple values or set of values of the hyper-parameters. Comparison is a must in this section. Also show convergence graphs for train, validation/test sets. The link to the code in colab / githut must be here. 
(\textbf{4 marks})
\section{Conclusion}
In this section you must again summarize your work. Tell about the limitations and possible future work. (\textbf{1 marks})

\noindent
\textbf{Count the total marks, it should be 25.}

\bibliography{myreferences}
\bibliographystyle{unsrt}
\end{document}
